\section*{Aufgaben}


\subsection*{Multiple Choice}

Seien $A_1, A_2, A_3 \in \F$ paarweise unabhängige Ereignisse, welche Aussage ist korrekt?
\begin{itemize}
	\item[\marked] Die Ereignisse $A_1, A_2, A_3$ sind nicht zwangsläufig unabhängig
	\item[$\square$] Die Ereignisse $A_1, A_2, A_3$ sind zwangsläufig unabhängig
\end{itemize}

\hrulefill

Es gilt $\Pm[X > t + s \; | \; X > s] = \Pm[X > t]$ für alle $t,s \geq 0$, falls
\begin{itemize}
	\item[$\square$] $X \sim \mathcal{U}([a,b])$
	\item[$\square$] $X \sim \text{Poisson}(\lambda)$
	\item[\marked] $X \sim \text{Exp}(\lambda)$ (Gedächnislosigkeit)
\end{itemize}

\hrulefill

Seien $X,Y$ zwei ZV mit gemeinsamer Dichte $f_{X,Y}$. Welche Aussage ist korrekt?
\begin{itemize}
	\item[\marked] $X,Y$ sind immer stetig
	\item[$\square$] Die ZV sind nicht notwendigerweise stetig.
\end{itemize}

\hrulefill

Seien $(X_i)_{i = 1}^n$ uiv. ZV mit Verteilungsfunktion $F_{X_i} = F$. Was ist die Verteilungsfunktion von $M = \text{max}(X_1,...,X_n)$?
\begin{itemize}
	\item[\marked] $F_M(a) = F(a)^n$
	\item[$\square$] $F_M(a) = 1 - F(a)^n$
	\item[$\square$] $F_M(a) = (1 - F(a))^n$
\end{itemize}

\hrulefill

Wenn das Niveau $\alpha$ eines Test kleiner wird
\begin{itemize}
	\item[\marked] Wird die Macht des Tests kleiner.
	\item[$\square$] WIrd die Macht des Tests grösser.
	\item[$\square$] Bleibt die Macht des Tests i.A. davon unbeeinflusst.
\end{itemize}

\subsection*{Sonstige Aufgaben}

\begin{comment}
	\textbf{Aufgabe}
	Es werden ein blauer und ein grüner Würfel geworfen. Wir wählen $\Omega = \{1,...,6\}^2$. Wir betrachten die Algebra:
	$$\F_{\text{sym}} = \{ A \subseteq \Omega \; | \; \forall (\omega_1, \omega_2) \in \Omega, (\omega_1, \omega_2) \in A \Leftrightarrow (\omega_2, \omega_1) \in A  \}$$
	Zeige, dass $\F_{\text{sym}}$ eine $\sigma$-Algebra ist. \medskip


	Wir müssen 3 Eigenschaften überprüfen. 
	\begin{enumerate}
		\item $\forall (\omega_1, \omega_2) \in \Omega \Leftrightarrow (\omega_2, \omega_1) \in \Omega$ daher gilt $\Omega \in \F_{\text{sym}}$.
		\item Sei $A \in \F_{\text{sym}}$. Somit gilt für jedes $(\omega_1, \omega_2) \in \Omega$
			$$(\omega_1, \omega_2) \in A \Leftrightarrow (\omega_2, \omega_1) \in \Omega$$
			was äquivalent ist zu
			$$(\omega_1, \omega_2) \in A^\complement \Leftrightarrow (\omega_2, \omega_1) \in A^\complement$$
			somit ist $A^\complement \in \F_{\text{sym}}$.
		\item Seien $A_1, A_2, ... \in \F_{\text{sym}}$ gilt:
			\begin{align*}
				(\omega_1, \omega_2) \in \cup_{i=1}^\infty A_i &\Leftrightarrow \exists i. (\omega_1, \omega_2) \in A_i \\
				&\Leftrightarrow \exists i. (\omega_2, \omega_1) \in A_i \\
				&\Leftrightarrow (\omega_2, \omega_1) \in \cup_{i=1}^\infty A_i
			\end{align*}
			Somit folgt $\cup_{i=1}^\infty A_i \in \F_{\text{sym}}$.
	\end{enumerate}

	\hrulefill
\end{comment}

\textbf{Aufgabe}
Sei $(X_i)_{i \geq 1}$ eine unendliche Folge von unabhängig $\text{Ber}(1/2)$-verteilten ZV. Wir betrachten folgenden Algorithmus:

\begin{algorithmic}
	\State $i \gets 1$
	\While{$X_i = X_{i+1}=1$}
		\State $i = i + 2$
	\EndWhile
	\State \Return $Z = X_i + 2 \cdot X_{i+1}$
\end{algorithmic}

Zeige, dass der Algorithmus immer nach endlich vielen Schritten terminiert (1). Zeige, dass $Z$ eine gleichverteilte ZV in $\{0,1,2\}$ ist (2). Konstruiert einen Algorithmus, der eine $\text{Ber}(1/5)$-verteilten ZV ausgibt (3). \medskip

(1) Wir definieren $A_j := \{ \text{While-Schlaufe wir j-Mal durchlaufen} \}$ und berechnen:
\begin{align*}
	\Pm[A_j] &= \Pm \left[ \bigcap_{i=1}^{2j} \{X_i = 1\} \cap (\{X_{2j+1} = 0\} \cup \{X_{2j+2} = 0\}) \right] \\
	&= \left( \Pi_{i=1}^{2j} \Pm[X_i = 1] \right) \cdot \Pm[X_{2j+1} = 0] \cdot \Pm[X_{2j+2} = 0] \\
	&= \left(\frac{1}{2}\right)^{2j} \cdot \frac{3}{4}
\end{align*}
Wenn wir nun über alle $A_j$ summieren, sehen wir, dass der Algorithmus immer in endlich Schritten terminieren wird. \medskip

(2) Wir wissen, dass alle $A_j$ disjunkt sind.
\begin{align*}
	\Pm[Z=0] &= \Pm[\{Z=0\} \cap A] + \Pm[\{Z=0\} \cap A^\complement] = \Sum_{j=0}^\infty \Pm[\{Z=0\} \cap A_j] \\
	&= \Sum_{j=0}^\infty \left(\frac{1}{2}\right)^{2j + 2} = \frac{1}{4}\Sum_{j=0}^\infty \left(\frac{1}{4}\right)^{j} = \frac{1}{4} \cdot \frac{1}{1 - \frac{1}{4}} = \frac{1}{3}
\end{align*}
Dies können wir nun auch für $1,2$ machen uns sehen, dass $Z$ gleichverteilt sein muss.

(3) Wir betrachten folgenden Algorithmus:
\begin{algorithmic}
	\State $i \gets 1$
	\While{$X_i = X_{i+2}=1 \text{ or } X_{i+1}=X_{i+2}=1$}
		\State $i = i + 3$
	\EndWhile
	\State \Return $Z = X_i + 2 \cdot X_{i+1} + 4 \cdot X_{i+2} = 4 \; ? \; 1 : 0$
\end{algorithmic}
Es ist leicht wie in $(1), (2)$ zu zeigen, dass er alle Eigenschaften erfüllt.

\hrulefill

\textbf{Aufgabe}
Sei $T \sim \text{Exp}(\lambda)$. Berechne die Dichte von $T' = c \cdot T^2$ und den Erwartungswert von $T'$. \medskip

Sei $\phi: \R \mapsto \R$ messbar und beschränkt. Wir definieren $\psi(x) = \phi(c \cdot x^2)$. Somit erhalten wir:
\begin{align*}
	\E[\phi(T')] &= \E[\phi(c \cdot T^2)] = \E[\psi(T)] = \int_{-\infty}^\infty \psi(x) \lambda e^{-\lambda x}\1_{x \geq 0}dx \\
	&= \int_0^\infty \phi(c \cdot x^2) \lambda e^{-\lambda x}dx = \int_0^\infty \phi(y) \lambda e^{-\lambda \sqrt{y / c}} \frac{dy}{2 \sqrt{cy}} \\
\end{align*}
Wobei wir die Dichte der Exponentialverteilung verwendet haben. daraus folgt:
$$f_{T'}(y) = \frac{\lambda}{2 \sqrt{cy}} e^{-\lambda \sqrt{y / c}}$$
Für den Erwartungswert gilt $\E[c \cdot T^2] = c \cdot \E[T^2]$:
$$\E[T^2] = \int_{0}^\infty x^2 \lambda e^{-\lambda x}dx = \frac{2}{\lambda^2}$$
Somit erhalten wir $\E[T'] = \frac{2c}{\lambda^2}$.

\hrulefill

\textbf{Aufgabe}
Konstruiere aus $U \sim \mathcal U[0,1]$ ein Ber$(1/2)$ verteilte ZV $Z$. \smallskip

Die verallgemeinerte Inverse $F^{-1}$ einer Ber$(1/2)$ verteilte ZV ist definiert als:
$$F^{-1}(\alpha) = \begin{cases}
	0, & 0 < \alpha < 2/3 \\
	1, & 2/3 \alpha < 1
\end{cases}$$

Aus Theorem 2.12 folgt, dass $Z:=F^{-1}(U)$ Ber$(1/2)$ verteilt ist.

\hrulefill

\textbf{Aufgabe}
Seien $X,Y$ ZV mit gemeinsamer Dichte $f_{X,Y}(x,y) = 4 \frac{y}{x^3} \mathbb{I}_{0 < x \leq 1, 0 < y \leq x^2}$. Berechne $\E[X / Y]$. \medskip

\begin{align*}
	\E[X / Y] &= \int_{- \infty}^\infty \int_{- \infty}^\infty \frac{x}{y} f_{X,Y}(x,y) dy dx \\
	&= \int_0^1 \int_0^{x^2} \frac{x}{y} \cdot 4 \frac{y}{x^3} dydx = \int_0^1 \int_0^{x^2}\frac{4}{x^2} dydx \\
	&= \int_0^1 \frac{4}{x^2} \cdot x^2 dx = 4
\end{align*}

\hrulefill

\textbf{Aufgabe}
Sei $T \sim \text{Exp}(\lambda)$. Berechne die Dichte von $T' = c \cdot T^2$ und den Erwartungswert von $T'$. \medskip

Sei $\phi: \R \mapsto \R$ messbar und beschränkt. Wir definieren $\psi(x) = \phi(c \cdot x^2)$. Somit erhalten wir:
\begin{align*}
	\E[\phi(T')] &= \E[\phi(c \cdot T^2)] = \E[\psi(T)] = \int_{-\infty}^\infty \psi(x) \lambda e^{-\lambda x}\1_{x \geq 0}dx \\
	&= \int_0^\infty \phi(c \cdot x^2) \lambda e^{-\lambda x}dx = \int_0^\infty \phi(y) \lambda e^{-\lambda \sqrt{y / c}} \frac{dy}{2 \sqrt{cy}} \\
\end{align*}
Wobei wir die Dichte der Exponentialverteilung verwendet haben. daraus folgt:
$$f_{T'}(y) = \frac{\lambda}{2 \sqrt{cy}} e^{-\lambda \sqrt{y / c}}$$
Für den Erwartungswert gilt $\E[c \cdot T^2] = c \cdot \E[T^2]$:
$$\E[T^2] = \int_{0}^\infty x^2 \lambda e^{-\lambda x}dx = \frac{2}{\lambda^2}$$
Somit erhalten wir $\E[T'] = \frac{2c}{\lambda^2}$.

\hrulefill

\textbf{Aufgabe}
Sei $X$ eine ZV mit Dichte $f_X$ und sei $Y = e^X$. Was ist die Dichte von $f_Y(y), y>0$, von $Y$?
$$F_Y = \mathbb{P}[Y \leq y] = \mathbb{P}[e^X \leq y] = \mathbb{P}[X \leq \log y] = F_X(\log y)$$ 
$$\Rightarrow f_y(y) = \frac{d}{dy}F_Y(y) = \frac{f_X(\log y)}{y}$$

\hrulefill

\textbf{Aufgabe}
Seien $X_1, X_2$ unabhängige ZV, beide gleichverteilt auf dem Interval $[0,1]$ und sei $X = \max (X_1, X_2)$. Berechne die Dichtefunktion von $X$ und $\mathbb{P}[X_1 \leq x \; | \; X \geq y]$.
$$F_X(t) = \mathbb{P}[\max(X_1, X_2) \leq t] = \mathbb{P}[X_1 \leq t] \cdot \mathbb{P}[X_2 \leq t] = F_{X_1}(t) \cdot F_{X_2}(t)$$
$$\Rightarrow f_X(t) = \frac{d}{dt} F_{X_1}(t) \cdot F_{X_2}(t) = \frac{d}{dt} t^2 \cdot \mathbb{I}_{0 \leq t \leq 1} = 2t \cdot \mathbb{I}_{0 \leq t \leq 1}$$

Für die Wahrscheinlichkeit brauchen wir eine Fallunterscheidung: \smallskip

$x < 0$ oder $1 < x$:
$$\mathbb{P}[X_1 \leq x \; | \; X \geq y] = 0$$

$0 \leq x \leq y \leq 1$:
$$\mathbb{P}[X_1 \leq x \; | \; X \geq y] = \frac{\mathbb{P}[X_1 \leq x \cap X \geq y]}{\mathbb{P}[X \geq y]} = \frac{x(1-y)}{1 - y^2}$$

$0 \leq y \leq x \leq 1$:
$$\mathbb{P}[X_1 \leq x \; | \; X \geq y] = \frac{\mathbb{P}[X_1 \leq x \cap X \geq y]}{\mathbb{P}[X \geq y]} = \frac{x - y^2}{1 - y^2}$$

\hrulefill

\textbf{Aufgabe}
Seien $X,Y$ diskrete ZV mit Gewichtsfunktion:
$$p(j,k) = \begin{cases}
	C \cdot (\frac{1}{2})^k & \text{für } k = 2,3,... \text{ und } j = 1,...,k-1  \\
	0 & \text{sonst}
\end{cases}$$

Bestimme die Konstante C.
\begin{align*}
	1 &\stackrel{!}{=} \Sum_{k=2}^\infty \Sum_{j=1}^{k-1} C \cdot \left( \frac{1}{2} \right)^k = C \cdot \Sum_{j=1}^\infty \left( \frac{1}{2} \right)^{j+1} \Sum_{k=0}^\infty \left( \frac{1}{2} \right)^k \\
	&= C \cdot \Sum_{j=1}^\infty \frac{\left( \frac{1}{2} \right)^{j+1}}{1 - \frac{1}{2}} = C \cdot \Sum_{j=1}^\infty \left( \frac{1}{2} \right)^j = C \cdot \frac{\frac{1}{2}}{1 - \frac{1}{2}} = C
\end{align*}

Beweise, dass bedingt auf das Ereignis $Y = k$ die ZV $X$ gleichverteilt auf $\{ 1, ..., k-1 \}$ ist.
$$p_Y(k) = \Sum_{j=1}^{k-1} \left( \frac{1}{2} \right)^k = (k-1) \cdot \left( \frac{1}{2} \right)^k $$

Nun verwenden wir den Satz der bedingten Wahrscheinlichkeit:
$$p_{X|Y}(j, k) = \frac{p(j,k)}{p_Y(k)} = \frac{\left( \frac{1}{2} \right)^2}{(k-1) \cdot \left( \frac{1}{2} \right)^2} = \frac{1}{1 - k} \quad \text{für } j = 1,...,k-1$$
Somit ist $X$ gegeben $Y = k$ gleichverteilt auf $\{ 1, ..., k-1 \}$.

\hrulefill

\textbf{Aufgabe}
Sei $X$ eine ZV mit Verteilungsfunktion $F_X$. Zeige, dass $X$ diskret ist.
$$ F_X(a) = \begin{cases}
	0, 		& a < 1 \\
	1/5, 	& 1 \leq a < 4 \\
	3/4, 	& 4 \leq a < 6 \\
	1, 		& 6 \leq a \\
\end{cases}$$ \medskip

Wir stellen fest, dass $\Pm[X = x] = 0$ für alle $x \notin \{1,4,6\}$. Da die Menge $\{1,4,6\}$ endlich ist, ist die ZV diskret.

\hrulefill

\textbf{Aufgabe}
Sei $T$ eine ZV mit Verteilungsfunktion $F_T$. Zeige, dass $t$ stetig ist.
$$ F_T(a) = \begin{cases}
	0, 				& a < 0 \\
	1 - e^{-2a}, 	& a < \geq 0
\end{cases}$$ \medskip

Wir stellen fest, dass $F_T$ stückweise stetig differenzierbar ist (auf $(-\infty, 0)$ und $(0, \infty)$). Somit folgt, dass $T$ eine stetige ZV ist.

\hrulefill

\textbf{Aufgabe}
Seien $X,Y$ ZV mit gemeinsamer Dichtefunktion:
$$f(x,y) = \frac{1}{x^2 y^2} \quad \text{für } x \geq 1, y \geq 1$$

Berechne die Verteilungsfunktion $F_U$ von $U = \frac{X}{Y}$. \medskip

Für $u \leq 0, F(u) = 0$. Für $u > 1$, ($X > Y$):
\begin{align*}
	F(u) &= \Pm[X/Y \leq u] = \Pm[X \leq Yu] = \int_1^\infty \int_1^{yu} \frac{1}{x^2 y^2} dxdy\\
	&= \int_1^\infty \frac{1}{y^2} - \frac{1}{uy^3} dy = 1 - \frac{1}{2u}
\end{align*}

Für $0 < u \leq 1$, ($X \leq Y$) gehen wir gleich vor, aber mit der unteren Integralgrenze $1/u$ für das äussere Integral. Wir erhalten dann $F(u) = \frac{u}{2}$.

\hrulefill

\textbf{Aufgabe}
Die ZV $X$ hat Verteilungsfunktion $F_\alpha(x) = \exp (-\exp (- (x - \alpha)))$. Zeige, dass $F_\alpha$ eine Verteilungsfunktion ist. \smallskip

Für $F_\alpha$ muss folgendes gelten:
\begin{itemize}
	\item (rechts)stetigkeit
	\item monoton wachsend
	\item $\lim_{x \to \infty} F_\alpha(x) = 1$
	\item $\lim_{x \to -\infty} F_\alpha(x) = 0$
\end{itemize}

Die Funktion ist stetig da sowohl $x - \alpha$ als auch $e^{-x}$ stetige Funktionen sind und $f,g$ stetig $\Rightarrow g \circ f$ stetig. Für die Monotonie verwenden wir: $x$ monoton wachsend $\Rightarrow e^x$ monoton wachsend. $x - \alpha$ ist monoton wachsend $\Rightarrow e^{-(x- \alpha)}$ ist monoton fallend $\Rightarrow \exp (-\exp (- (x - \alpha)))$ ist monoton wachsend. Zuletzt gilt noch:
$$\lim_{x \to -\infty} e^{- e^{-(x-\alpha)}} = \lim_{x \to \infty} e^{-x} = 0, \qquad \lim_{x \to \infty} e^{- e^{-(x-\alpha)}} = \lim_{x \to \infty} e^0 = 1$$

\hrulefill

\textbf{Aufgabe}
Seien $X \sim \mathcal P (\lambda), Y \sim \mathcal P (\mu)$, zeige, dass $X + Y \sim \mathcal P (\lambda + \mu)$.

Für $k \in \N_0$:
\begin{align*}
	\Pm[X + Y = k] &= \sum_{l = 0}^k \Pm[X + Y = k, Y = l] = \sum_{l = 0}^k \Pm[X = k - l, Y = l] \\
	&= \sum_{l = 0}^k \frac{\lambda^{k-l}}{(k-l)!}e^{-\lambda} \frac{\mu^l}{l!}e^{-\mu} = e^{-(\lambda + \mu)} \sum_{l = 0}^k \frac{\lambda^{k-l}}{(k-l)!}\frac{\mu^l}{l!} \\
	&= e^{-(\lambda + \mu)} \frac{1}{k!} \sum_{l = 0}^k \frac{k!}{(k-l)! \cdot l!} \lambda^{k-l} \mu^l = e^{-(\lambda + \mu)} \frac{(\lambda + \mu)^k}{k!}
\end{align*}

\hrulefill

\textbf{Aufgabe}
Ein Gerät hat eine Lebensdauer $H \sim \text{Exp}(\lambda)$ mit Erwartungswert 60. \smallskip

1) Nun haben wir zwei Geräte und wollen wissen wie die Verteilung für die Zeit bis zum ersten Defekt verteilt ist. 

Aus dem Erwartungswert ergibt sich $H_1, H_2 \sim \text{Exp}(1 / 60)$. Nun ist $T = \min (H_1, H_2)$ mit Verteilungsfunktion:
\begin{align*}
	F_T(t) &= \Pm[T \leq t] = \Pm[\min (H_1, H_2) \leq t] \\
	&= 1 - \Pm[\min (H_1, H_2) > t] = 1 - \Pm[H_1 > t, H_2 > t] \\
	&= 1 - \Pm[H_1 > t] \cdot \Pm[H_2 > t] \\
	&= 1 - \exp(-2 \lambda t)
\end{align*}

D.h. $T \sim \text{Exp}(1/30)$.

\smallskip

2) Wir ersetzen beide Geräte wenn eines Defekt ist und wollen wissen wie gross die Wahrscheinlichkeit ist, mehr als 35 Ersatzteile in drei Jahren zu benötigen.

Sei $T_i$ die Zeit, bis das $i$-te Teil ersetzt wird, $S = T_1 + ... + T_{36}$. Nun wollen wir mit dem Grenzwertsatz die Wahrscheinlichkeit berechnen.
\begin{align*}
	\Pm[S \leq 1095] &= \Pm \left[ \frac{S - n \E[T_i]}{\sqrt{\sigma_T^2 n}} \leq \frac{1095 - n \E[T_i]}{\sqrt{\sigma_T^2 n}}\right] \\
	&= \Pm \left[ \frac{S - 36 \cdot 30}{\sqrt{30^2 \cdot 36}} \leq \frac{1095 - 36 \cdot 30}{\sqrt{30^2 \cdot 36}} \right] \\
	&= \Pm \left[ \frac{S - 1080}{180} \leq \frac{15}{180} \right] \\
	&\approx \Phi(1/12) \approx \Phi(0.08) = 0.5319
\end{align*}

\hrulefill

\textbf{Aufgabe}
Um die Anzahl Fische $N$ in einem See zu bestimmen gehen wir wie folgt vor, zuerst werden 500 Fische gefangen und markiert. Danach werden wieder 200 Fische gefangen und die Anzahl $X$ der markierten Fische gezählt. \smallskip

1) $X \sim \text{Bin}(n, \theta)$, wie gross ist $n$? Wie gross ist $\theta$, wenn die Gesamtzahl der Fische $N = 2000$ ist?

$n = 200$, da wir 200 Fische herausziehen. $\theta = \frac{500}{N} = \frac{500}{2000} = \frac{1}{4}$ \smallskip

2) Die Beobachtung gibt einen Wert für $X$ von 40. Gebe eine Schätzung für $\theta$ und eine Schätzung für $N$ ab.

Wir schätzen $\theta$ mit $T = X / n$, der realisierte Schätzwert ist also $\theta = 1/5$. Wenn wir nun $\theta = 500 / N$ nach $N$ auflösen erhalten wir $N = 2500$. \smallskip

3) Bestimme ein approximatives Konfidenzintervall für $\theta$ mit $\alpha = 0.05$.

$$T = \frac{X - n \theta}{\sqrt{n\theta(1-\theta)}} \sim \mathcal N (0,1)$$

Aus dem zentralen Grenzwertsatz folgt daher $\Pm_\theta [-1.96 \leq T \leq 1.96] \geq 0.95$. Unter verwendung von $\theta(1- \theta) \leq 1/4$ ergibt sich ein approximatives Vertrauensintervall für $\theta$ von:
$$[T - \frac{1.96}{2\sqrt{n}}, T + \frac{1.96}{2 \sqrt{n}}] = [0.13, 0.27]$$

\hrulefill

\textbf{Aufgabe}
Wir haben eine Münze und vermuten, dass Sie gezinkt ist und häufiger auf Kopf landet. Wir bezeichnen den $i$-ten Wurf mit $X_i$ und das Resultat ist 1 wenn Kopf und 0 wenn Zahl. Wir beobachten folgende Ergebnisse:
$$[0,1,1,1,1,1,0,1,1,1]$$

Führe einen Test mit $\alpha = 0.01$ durch. \smallskip

1) Modell: \quad Wir wählen $X_i$ uiv. Ber$(\theta)$ unter $\Pm_\theta$ wobei $\theta \in [0,1]$. \smallskip

2) Nullhypothese und Alternativhypothese:

$$H_0: \theta = \frac{1}{2} \qquad H_A: \theta > \theta_0$$

3) Teststatistik:

Für $X \sim \text{Ber}(\theta)$ ist die Gewichtsfunktion $p_X(k) = p^k(1-p)^{1-k}$. Somit ist der Likelihood-Quotient:
\begin{align*}
	R(x_1,...,x_10; \theta_A, \theta_0) &= \frac{L(x_1,...,x_n; \theta_A)}{L(x_1,...,x_n; \theta_0)} \\
	&= \left( \underbrace{\frac{\theta_A}{\theta_0}}_{> 1} \right)^{\sum x_i} \left( \underbrace{\frac{1 - \theta_A}{1 - \theta_0}}_{< 1} \right)^{\sum x_i}
\end{align*}
 Somit ist $R$ genau dann gross, wenn $\sum_{i=1}^{10} X_i$ gross ist. Daher wählen wir die Teststatistik:
 $$T = \sum_{i=1}^{10}X_i$$

4) Verteilung der Teststatistik unter $H_0$:

Da $X_i \sim \text{Ber}(\theta)$ ist $T \sim \text{Bin}(10, \theta)$ unter $\Pm_\theta$. \smallskip

5) Verwerfungsbereich:

Wir wählen $K = (c, 10]$. Um $c$ zu bestimmen rechnen wir:
$$\Pm_\theta[T \in K] \leq \alpha \Rightarrow \Pm_\theta[T \leq c] \geq 1- \alpha$$

Mit der gegebenen Tabelle sehen wir, dass $c = 9$ der kleinste Wert für $c$ ist, welcher die Bedingung erfüllt. Somit ist $K = (9,10] = \{10\}$. \smallskip

6) beobachte Wert der Teststatistik: \quad $T(\omega) = 8$ \smallskip

7) Testentscheid: \quad $T(\omega) \notin K$, wir verwerfen die Nullhypothese nicht.

\hrulefill


Die Höchsttemperaturen in den ersten 9 Tagen im April 2020 waren
\resizebox{\columnwidth}{!}{
\begin{tabular}{|c|ccccccccc|}
\hline Tag $i$ & 1 & 2 & 3 & 4 & 5 & 6 & 7 & 8 & 9 \\
\hline Temperatur $x_{i}$ & 15 & $18.4$ & $20.9$ & 16 & 17 & 23 & $21.1$ & 21 & 15 \\
\hline
\end{tabular}
}
Nehmen Sie an, dass die Höchsttemperaturen Realisierungen von unabhängigen und je $\mathcal{N}\left(\mu, \sigma^{2}\right)$ verteilten Zufallsvariablen sind, wobei $\mu$ und $\sigma^{2}>0$ unbekannt sind. Im April 2019 lag die durchschnittliche Höchsttemperatur bei 22 Grad. Wir möchten auf dem 5\%-Niveau testen, ob durch den Flugstopp im April 2020 die erwartete tägliche Höchsttemperatur im Vergleich zum Wert vom Vorjahr gesunken ist.

(a) Führen Sie einen geeigneten Test durch. Geben Sie dazu
\begin{enumerate}[label=\roman*]
	\item das Modell,
	\item die Hypothese und Alternative,
	\item die Teststatistik,
	\item die Verteilung der Teststatistik unter der Hypothese,
	\item den Verwerfungsbereich,
	\item den beobachteten Wert der Teststatistik, sowie
	\item den Testentscheid an.
\end{enumerate}
Kennzahlen: $\bar{x}_{9}=18.6, s_{9}=3.0, s_{9}^{2}=9.0$.

\textbf{Lösung:}
\begin{enumerate}[label=\roman*]
	\item Sei $X_{1}, \ldots, X_{9}$ die Stichprobe, welche die Daten $x_{1}, \ldots, x_{9}$ realisiert. Nach der Aufgabenstellung sind $X_{1}, \ldots, X_{9}$ i.i.d. $\sim \mathcal{N}\left(\mu, \sigma^{2}\right)$ unter $P_{\vartheta}$, wobei $\vartheta=\left(\mu, \sigma^{2}\right)$ ein unbekannter Parameter ist.
	\item Es ist naheliegend, als Hypothese und Alternative
$$
H_{0}: \mu=\mu_{0}:=22 \text { und } H_{A}: \mu<\mu_{0}
$$
zu wählen.
	\item Da $\mu$ und $\sigma^{2}$ unbekannt sind, ist es naheliegend, einen $t$-Test durchzuführen. Als Teststatistik wählen wir also
$$
T=\frac{\bar{X}_{9}-\mu_{0}}{S_{9} / \sqrt{9}} .
$$
wobei $\bar{X}_{n}=\frac{1}{n} \sum_{i=1}^{n} X_{i}$ und $S_{n}^{2}=\frac{1}{n-1} \sum_{i=1}^{n}\left(X_{i}-\bar{X}_{n}\right)^{2}$.
	\item Unter $H_{0}$ folgt $T$ einer $t$-Verteilung mit 8 Freiheitsgraden.
	\item Nach der Alternative hat der kritische Bereich die Form $K_{<}=\left(-\infty, c_{<}\right)$für ein zu bestimmendes $c_{<}$. Für $\alpha=0.05$ wählen wir $c_{<}$so, dass
	$$
	\alpha=P_{H_{0}}\left[T<c_{<}\right] .
	$$
	Also ist $c_{<}=t_{n-1, \alpha}=-t_{n-1,1-\alpha}=-t_{8,0.95}=-1.860$.
	\item Der beobachtete Wert der Teststatistik ist
	$$
	T(\omega)=t\left(x_{1}, \ldots, x_{9}\right)=\frac{18.6-22}{3 / 3}=-3.4 .
	$$
	\item Wegen $T(\omega) \in K_{<}$verwerfen wir somit die Hypothese und nehmen die Alternative an. Die Daten sprechen also tatsächlich dafür, dass die durchschnittliche Höchsttemperatur gesunken ist.
\end{enumerate}
(b) Bestimmen Sie mit den vorhandenen Tabellen eine möglichst scharfe obere Grenze für den realisierten p-Wert.

\textbf{Lösung:}
Der realisierte p-Wert ist
\begin{align*}
	\mathrm{p}-\operatorname{Wert}(\omega) & =\left.P_{H_{0}}\left[T<t_{0}\right]\right|_{t_{0}=T(\omega)}\\ & =P_{H_{0}}[T<-3.4]=1-P_{H_{0}}[T<3.4] .
\end{align*}
Aus den vorhandenen Tabellen folgt wegen $t_{8,0.995}=3.355<3.4$, dass
$$
0.995 \leq P_{H_{0}}[T<3.4] \leq 1 .
$$
Also gilt
$$
0 \leq \mathrm{p}-\operatorname{Wert}(\omega) \leq 0.005
$$
\hrulefill

\textbf{Aufgabe}
(d) Nehmen Sie an, die wirkliche mittlere Trocknungszeit beträgt 85 Minuten. Be-rechnen Sie die Wahrscheinlichkeit eines Fehlers 2. Art für diese Alternative.
[$K = (-\infty, -2.33)$]
Ein Fehler 2. Art für die Alternative $\mu_A = 85$ [Minuten] tritt auf, falls die Nullhypothese nicht verworfen wird, obwohl die mittlere Trocknungszeit in Wirklichkeit 85 Minuten ist. Die Wahrscheinlichkeit für so einen Fehler ist
\begin{align*}
	\Pm_{\mu_A}[T \notin K] & = \Pm_{\mu_A}[\frac{\overline{X}_n - \mu_0}{\sigma / \sqrt{n}}] \\
	& = \Pm_{\mu_A}[\frac{\overline{X}_n - \mu_A}{\sigma / \sqrt{n}} - \frac{\mu_0 - \mu_A}{\sigma / \sqrt{n}} > -2.33] \\
	& = \Pm_{\mu_A}[\frac{\overline{X}_n - \mu_A}{\sigma / \sqrt{n}} - > \ldots] \\
	& = \ldots
\end{align*}


\hrulefill

\textbf{Aufgabe}
Seien $U_1, U_2, U_3$ unabhängige $\mathcal U[0,1]$ verteilte ZV. Wir betrachten die stetigen ZV:
$$L = \min(U_1, U_2, U_3), \qquad M = \max(U_1, U_2, U_3)$$

1) Berechne die Dichte von $M$ und $L$.

Die drei ZV haben die gemeinsame Dichte: 
$$f(u_1, u_2, u_3) = \mathbb I_{u_1 \in [0,1]} \mathbb I_{u_2 \in [0,1]} \mathbb I_{u_3 \in [0,1]}$$

Sei nun also $\phi$ stückweise stetig und beschränkt. Wir berechnen:
$$\E[\phi(M)] = \int_0^1 \int_0^1 \int_0^1 \phi(\max(u_1, u_2, u_3)) du_1 du_2 du_3$$

Wir unterscheiden 6 verschiedene Fälle, je nachdem welche Variable das Maximum annimmt. Wir berechnen den Fall $u_3 < u_2 < u_1$ und erhalten:
$$\int_{-\infty}^\infty \phi(u_1) \frac{1}{2} u_1^2 \mathbb I_{u_1 \in [0,1]} du_1$$

Da die sechs Fälle symmetrisch sind haben wir
$$\E[\phi(M)] = 6 \cdot \int_{-\infty}^\infty \phi(m) \frac{1}{2} m^2 \mathbb I_{m \in [0,1]} dm$$

somit ist die Dicht $f_M(m) = 3m^2 \mathbb I_{m \in [0,1]}$. Da $(1 - U_1, 1- U_2,1- U_3)$ aus Symmetriegründen die gleiche gemeinsame Dichte hat wie $(U_1, U_2, U_3)$, hat L die gleiche Dichte wie $1- M$ und auf diese Weise erhält man ebenfalls $f_L(l) = 3(1 - l)^2 \mathbb I_{l \in [0,1]}$. \smallskip

2) Zeige, dass für $\varphi, \phi$ stückweise stetig, beschränkt
$$\E[\phi(M) \varphi(L)] = \int_{-\infty}^\infty \int_{-\infty}^\infty \phi(m) \varphi(l) 6(m-l) \mathbb I_{0 \leq l \leq m \leq 1} dldm$$

Wir verwenden das Resultat aus der ersten Teilaufgabe und die Symmetrie:
\begin{align*}
	\int_0^1 & \int_0^1 \int_0^1 \phi(\max(u_1, u_2, u_3)) \varphi(\max(u_1, u_2, u_3)) du_1 du_2 du_3 \\
	&= \int_0^1 \int_0^1 \phi(u_1) \varphi(u_3) (u_1 - u_3) \mathbb I_{u_3 \leq u_1} du_1 du_3 \\
	\Rightarrow \E[\phi(M) \varphi(L)] &= 6 \cdot \int_{-\infty}^\infty \int_{-\infty}^\infty \phi(m) \varphi(l) (m-l) \mathbb I_{0 \leq l \leq m \leq 1} dldm
\end{align*}

3) Bestimme die gemeinsame Dicht und Verteilungsfunktion von $(M, L)$.

Wir bestimmen $\phi(x) = \mathbb I_{x \leq a}, \varphi(x) = \mathbb I_{x \leq b}$ und erhalten::
\begin{align*}
	F_{M,L}(a,b) &= \Pm[M \leq a, L \leq b] = \E[I_{M \leq a} I_{L \leq b}] \\
	&= \int_{-\infty}^a \int_{-\infty}^b 6(m-l) \mathbb I_{0 \leq l \leq m \leq 1} dmdl
\end{align*}

Somit ist $f_{M,L}(m,l) = 6(m-l) \mathbb I_{0 \leq l \leq m \leq 1}$ die gemeinsame Dicht. Für die Verteilungsfunktion berechnen wir das obige Integral, hierbei unterscheiden wir verschiedene Fälle:

$a \leq 0$ oder $b \leq 0$:
$$F_{M,L}(a,b) = 0$$

$a \geq 1$:
$$F_{M,L}(a,b) = F_L(b) = 1 - (1-b)^3$$

$b \geq 1$:
$$F_{M,L}(a,b) = F_M(a) = a^3$$

$0 \leq a \leq b \leq 1$:
$$F_{M,L}(a,b) = \Pm[M \leq a, L \leq b] = \Pm[M \leq a] = a^3$$

$0 \leq b \leq a \leq 1$:
\begin{align*}
	\Pm[M \leq a, L \leq b] &= \int_{-\infty}^a \int_{-\infty}^b 6(m-l) \mathbb I_{0 \leq l \leq m \leq 1} dmdl \\
	&= \int_0^a \int_0^{\min(b, m)} 6(m-l) dldm \\
	&= \int_0^a [6ml - 3l^2]_0^{\min(b, m)} dm \\
	&= \int_0^b 3m^2 dm + \int_b^a 3b(2m - b) dm \\
	&= b^3 + 3ab(a-b)
\end{align*}

\hrulefill