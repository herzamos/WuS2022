\begin{center}
	\Large{\textbf{Statistik}}
\end{center}

\section*{Schätzer}

Wir nehmen an, dass wir wie einen Wahrscheinlichkeitsraum $(\Omega, \F, \Pm)$ und Zufallsvariablen $X_i$ haben. Zudem haben wir einen Parameterraum $\Theta \subset \R$, wobei $(\Pm_\theta)_{\theta \in \Theta}$ eine Familie von Wahrscheinlichkeitsmassen ist. $\Pm_\theta$ wird auch Modell genannt. Die Gesamtheit der beobachteten Daten nennen wir Stichprobe und die Anzahl $n$ den Stichprobenumfang.

\begin{mainbox}{Schätzer}
    Ein \textbf{Schätzer} ist eine Zufallsvariable $T: \Omega \mapsto \R$, von der Form:
    $$T = t(X_1,...,X_n), \quad t:\R^n \mapsto \R$$
\end{mainbox}

Ein Schätzer $T$ heisst \textbf{erwartungstreu} für den Modellparameter, falls für alle $\theta \in \Theta$ gilt:
$$\E_\theta[T] = \theta$$

Der \textbf{Bias} (erwarteter Schätzfehler) von $T$ im Modell $\Pm_\theta$ ist definiert als:
$$\E_\theta[T] - \theta$$

Der \textbf{mittlere quadratische Schätzfehler (MSE)} von $T$ im Modell $\Pm_\theta$ ist definiert als:
$$\text{MSE}_\theta[T] = \E_\theta[(T-\theta)^2]$$

$$\text{MSE}_\theta[T] = \text{Var}_\theta(T) + (\E_\theta[T] - \theta)^2$$


%------------
% \columnbreak
%------------



\Title{Maximum-Likelihood-Methode}

Nehmen wir an $X_1,...,X_n \in W$ sind ZV unter $\Pm_\theta$. Die \textbf{Likelikhood Funktion} ist definiert durch:
\begin{mainbox}{Likelikhood Funktion}
    $$L(x_1, ..., x_n; \theta) = \begin{cases}
        p(x_1, ..., x_n; \theta) & \text{im diskreten Fall} \\
        f(x_1, ..., x_n; \theta) & \text{im stetigen Fall} \\
        \end{cases}$$
\end{mainbox}

wobei $p$ respektive $f$ die gemeinsame Gewichtsfunktion / Dichtefunktion ist. Falls die $X_i$ uiv. sind unter $\Pm_\theta$, so gilt:
$$p_X(x_1,...,x_n;\theta) = \prod_{i=1}^n p_{X_i}(x_i; \theta)$$ $$f_X(x_1,...,x_n;\theta) = \prod_{i=1}^n f_{X_i}(x_i; \theta)$$

Für jedes $x_1, ..., x_n \in W$, sei $t_{\text{ML}}(x_1, ..., x_n)$ der Wert, der die Funktion $\theta \mapsto L(x_1, ..., x_n; \theta)$ maximiert. Ein \textbf{Maximum-Likelihood-Schätzer} ist definiert als: 
$$T_{\text{ML}} = t_{\text{ML}}(X_1,...,X_n)$$

Die \textbf{Log-Likelihood} Funktion hat den Vorteil, dass sie durch eine Summe anstelle eines Produkts gegeben ist und daher oftmals einfach zu berechnen ist.


\section*{Konfidenzintervalle}

Wir haben nun Methoden für Schätzer von unbekannten Parameter kennengelernt. Nun wollen wir wissen wie weit diese Schätzer vom effektiven Wert $p$ weg liegen.

\begin{mainbox}{Konfidenzintervall}
    Sei $\alpha \in [0,1]$. Ein \textbf{Konfidenzintervall} für $\theta$ mit Niveau $1 - \alpha$ ist ein Zufallsintervall $I = [A, B]$, sodass gilt 
    $$\forall \theta \in \Theta. \quad \Pm_\theta[A \leq \theta \leq B] \geq 1 - \alpha$$
    wobei $A,B$ ZV der Form $A = a (X_1, ..., X_n)$ und $B = b (X_1, ..., X_n)$, mit $a,b : \R^n \mapsto \R$, sind.
\end{mainbox}

Wenn wir nun einen $T = T_{\text{ML}} \sim \mathcal{N}(m, 1/n)$ haben, (dz.Bsp. $T_{\text{ML}}$ mit $X_1,...,X_n$ uiv. $\mathcal{N}(m, 1)$ oder $T = T_{ML} = \frac{X_1 + \ldots + X_n}{n}$) suchen wir einen Konfidenzinterfall der Form: 
$$I = [T - c/\sqrt{n}, T + c / \sqrt{n}]$$

Hierbei gilt:
\begin{align*}\Pm_\theta[T - c/\sqrt{n} \leq m \leq T + c/\sqrt{n}] &= \Pm_\theta[-c \leq Z \leq c] \\ &= \Pm_\theta[Z \leq c] - \Pm_\theta[Z < -c] \\ &= \Pm_\theta[Z \leq c] - (1 - \Pm_\theta[Z \leq c]) \\ &= 2\phi(c) - 1\end{align*}

wobei $Z = \sqrt{n}(T-m) =  \frac{X_1 + \ldots + X_n - mn}{\sqrt{n}}$ eine standardnormalverteilte ZV ist.

\Title{Verteilungsaussagen}

\begin{mainbox}{{$\chi^2$-Verteilung}}
Eine stetige ZV heisst \textbf{$\chi^2$-verteilt} mit $m$ Freiheitsgrade, falls ihre Dichte gegeben ist durch
$$f_X(x) = \frac{1}{2^{\frac{m}{2}} \Gamma (\frac{m}{2})} x^{\frac{m}{2}-1} e^{-\frac{1}{2}x} \1_{x > 0},$$
\end{mainbox}
wobei 
$$\Gamma(v) = \int_0^\infty t^{v-1} e^{-t} dt.$$
Wir schreiben dann $X \sim \chi^2_m$.

Für natürliche Zahlen gilt $\Gamma(n) = (n-1)!$. Ein Spezialfall ist $m = 2$, hierbei erhalten wir $X \sim \text{Exp}(\frac{1}{2})$ \medskip

\begin{subbox}{} Für ZV $X_1, ..., X_m$ u.i.v. mit $X_i \sim \mathcal{N}(0,1)$ ist die Summe 
$$Y = \Sum_{i=1}^m X_i^2 \sim \chi_m^2$$
\end{subbox}
\begin{mainbox}{t-Verteilung}
Eine stetige ZV $X$ heisst \textbf{$t$-verteilt} mit $m$ Freiheitsgrade, falls ihre Dichte gegeben ist durch 
$$f_X(x) = \frac{\Gamma (\frac{m+1}{2})}{\sqrt{m \pi} \Gamma (\frac{m}{2})} \left( 1 + \frac{x^2}{m} \right)^{-\frac{m+1}{2}},$$
\end{mainbox}
\begin{subbox}{} Für $X, Y$ unabhängig mit $X \sim \mathcal{N}(0,1)$ und $Y \sim \chi^2_m$, ist der Quotient
$$Z = \frac{X}{\sqrt{\frac{Y}{m}}} \sim t_m.$$\end{subbox}


\Title{Normalverteilung mit $\mu, \sigma^2$ unbekannt }

\begin{subbox}{} Seien $X_1,...,X_n$ uiv. $\sim \mathcal{N}(\mu, \sigma^2)$. Dann sind
$$\overline{X}_n = \frac{1}{n} \Sum_{i=1}^n X_i \qquad \text{und} \qquad S^2 = \frac{1}{n-1} \Sum_{i = 1}^n (X_i - \overline{X}_n)^2$$
unabhängig. \end{subbox}

Nun haben wir $X_1,...,X_n$ ZV, die alle unter $\Pm_\theta$ uiv. $\sim \mathcal{N}(\mu, \sigma^2)$ sind. Die offensichtlichen Schätzer für $\mu, \sigma^2$ sind das Stichprobenmittel $\overline{X}_n$ und die Stichprobenvarianz $S^2$. Für jedes $\theta \in \Theta$ gilt:

$$\frac{\overline{X}_n - \mu}{S / \sqrt{n}} \sim t_{n-1} \text{ unter } \Pm_\theta$$

Also wollen wir:

$$1- \alpha \leq \Pm_\theta \left[ | \frac{\overline{X}_n - \mu}{S / \sqrt{n}} | \leq \frac{...}{ S / \sqrt{n}} \right]$$

Um eine kleines Intervall zu erhalten, wollen wir die Bedingung mit Gleichheit erfüllen und nehmen $\frac{...}{ S / \sqrt{n}} = t_{n-1, 1 - \alpha / 2}$, somit erhalten wir für $\mu$ folgendes Konfidenzintervall:

$$\left[ \overline{X}_n - t_{n-1, 1 - \alpha / 2} \frac{S}{\sqrt{n}},  \overline{X}_n + t_{n-1, 1 - \alpha / 2} \frac{S}{\sqrt{n}} \right]$$

Um ein Konfidenzintervall für $\sigma^2$ zu konstruieren brauchen wir:

$$\frac{1}{\sigma^2} (n-1) S^2 = \frac{1}{\sigma^2} \Sum_{i=1}^n (X_i - \overline{X}_n)^2 \sim \chi_{n-1}^2 \text{ unter } \Pm_\theta$$

Mit der Notation $\chi_{m, \gamma}^2$ für das $\gamma$-Quantil einer $\chi_m^2$ Verteilung wollen wir:

$$1 - \alpha = \Pm_\theta \left[ \chi_{n-1, \alpha/2}^2 \leq \frac{1}{\sigma^2} (n-1) S^2 \leq \chi_{n-1, 1- \alpha/2}^2\right]$$

Somit erhalten wir das Konfidenzintervall:

$$\left[ \frac{(n-1) S^2}{\chi_{n-1, 1-\alpha/2}^2} , \frac{(n-1) S^2}{\chi_{n-1, \alpha/2}^2} \right]$$

Fazit: Das wichtigste Tool zur Bestimmung von Konfidenzintervallen sind Verteilungsaussagen über Schätzer. Die ist im Allgemeinen aber schwierig / nicht möglich.


\Title{Approximative Konfidenzintervalle}

Zur Erinnerung der zentrale Grenzwertsatz besagt, dass wenn $X_i$ in $\Pm_\theta$ uiv. sind, dann ist $\sum_{i=1}^n X_i$ approximativ normalverteilt mit $\mu = n \E[X_i]$ und $\sigma^2 = n \text{Var}_\theta[X_i]$. Insbesondere können wir daraus auch eine standard normalverteilte ZV erhalten (siehe Grenzwertsatz).


\section*{Tests}


\Title{Null- und Alternativhypothese}

\begin{mainbox}{Null- und Alternativhypothese}
    Die \textbf{Nullhypothese $H_0$} und die \textbf{Alternativhypothese $H_A$} sind zwei Teilmengen $\Theta_0 \subseteq \Theta, \Theta_A \subseteq \Theta$ wobei $\Theta_0 \cap \Theta_A = \emptyset$. Eine Hypothese heisst \textbf{einfach}, falls die Teilmenge aus einem einzelnen Wert besteht; sonst zusammengesetzt.
\end{mainbox}


\Title{Test und Entscheidung}

\begin{mainbox}{Test}
    Ein \textbf{Test} ist ein $(T, K)$, wobei $T$ eine ZV der Form $T = t(X_1,...,X_n)$ ist und $K \subseteq \R$ ist eine deterministische Teilmenge von $\R$. Man nennt $T$ die \textbf{Teststatistik} und $K$ den \textbf{Verwerfungsbereich} oder kritischen Bereich.
\end{mainbox}

Wir wollen nun anhand der Daten $(X_1(\omega), ..., X_n(\omega))$ entscheiden ob die Nullhypothese akzeptiert oder verworfen wird. Dafür berechnen wir zuerst die Teststatistik $T(\omega) = t(X_1(\omega), ..., X_n(\omega))$ und gehen dann wie folgt vor:
\begin{itemize}
    \item die Hypothese $H_0$ wird verworfen, falls $T(\omega) \in K$
    \item die Hypothese $H_0$ wird nicht verworfen bzw. angenommen, falls $T(\omega) \notin K$
\end{itemize}

\begin{subbox}{Fehler 1. Art}
    Ein \textbf{Fehler 1. Art} ist, wenn die $H_0$ verworfen wird, obschon sie richtig ist.
    $$\Pm_\theta[T \in K], \quad \theta \in \Theta_0$$
\end{subbox}


\begin{subbox}{Fehler 2. Art}
    Ein \textbf{Fehler 2. Art} ist, wenn die $H_0$ akzeptiert wird, obschon sie falsch ist.
    $$\Pm_\theta[T \notin K] = 1 - \Pm_\theta [T \in K] \quad \theta \in \Theta_A$$

\end{subbox}


\Title{Signifikanzniveau und Macht}

Bei der Auswahl eines geeigneten Tests ist insbesondere die Minimierung von Fehlern 1. Art entscheidend. \smallskip

\begin{mainbox}{Signifikanzniveau}
    Sei $\alpha \in [0,1]$. Ein Test hat nun \textbf{Signifikanzniveau} $\alpha$ falls:
    $$\forall \theta \in \Theta_0. \quad \Pm_\theta[T \in K] \leq \alpha$$
\end{mainbox}

Das Sekundäre Ziel ist es den Fehler 2. Art zu vermeiden. \smallskip

\begin{mainbox}{Macht}
    Die \textbf{Macht} eines Tests wird definiert als Funktion:
    $$\beta: \Theta_A \mapsto [0,1], \quad \theta \mapsto \Pm_\theta[T \in K]$$
\end{mainbox}

\Bem $\alpha$ klein entspricht einer kleine Wahrscheinlichkeit für Fehler 1. Art, während $\beta$ gross einer kleinen Wahrscheinlichkeit für einen Fehler 2. Art entspricht. \smallskip

Das obige asymmetrische Vorgehen macht es schwieriger, die Nullhypothese zu verwerfen als sie beizubehalten. Ein guter Test wird deshalb als Nullhypothese immer die Negation der eigentlich gewünschten Aussage benutzen.

\Bem $\beta = 1 - \Pm_{H_A}[H_0 \text{ angenommen}]$. Das heisst, whenn zum Beispiel $\alpha = \Pm_{H_A}[T \leq 4]$ dann $\beta = 1 - \Pm[T > 4]$.

%------------
% \columnbreak
%------------



\Title{Konstruktion von Tests}

Wir nehmen an, dass $X_1,...,X_n$ diskret oder gemeinsam stetig sind unter $\Pm_{\theta_0}$ und $\Pm_{\theta_A}$, wobei $\theta_0 \neq \theta_A$ von der einfachen Form sind. \smallskip

\begin{subbox}{Likelihood-Quotient}
Der \textbf{Likelihood-Quotient} ist somit wohldefiniert:
$$R(x_1,...,x_n) = \frac{L(x_1,...,x_n; \theta_A)}{L(x_1,...,x_n; \theta_0)}$$
\end{subbox}

Wobei wir $R(x_1,...,x_n) = + \infty$ definieren falls $L(x_1,...,x_n; \theta_0) = 0$ sein sollte. Daraus ergibt sich, dass $R \gg 1 \Rightarrow H_A > H_0$ und $R \ll 1 \Rightarrow H_A < H_0$. \smallskip

\begin{mainbox}{Likelihood-Quotient-Test} 
    Sei $c \geq 0$. Der \textbf{Likelihood-Quotient-Test} (LQ-Test) mit Parameter $c$ ist definiert durch:
    $$ T = R(x_1,...,x_n) \quad \text{ und } \quad K = (c, \infty] $$
\end{mainbox}

Der LQ-Test ist optimal, da jeder andere Test mit einem kleineren Signifikanzniveau auch eine kleinere Macht hat:

\begin{mainbox}{Neyman-Paerson-Lemma}
    Sei $c \geq 0$ und $(T,K)$ der LQ-Test mit Parameter $c$ und $H_A = \theta_A, H_0 = \theta_0$.
$$\alpha^* = \Pm_{\theta_0}[T \in K]$$

Sei $(T', K')$ ein Test mit S-Niveau $\alpha \leq \alpha^*$ so gilt:
$$\Pm_{\theta_A}[T' \in K'] \leq \Pm_{\theta_A}[T \in K]$$
\end{mainbox}

\Title{p-Wert}

Wir wollen eine Hypothese $H_0 : \theta = \theta_0$ gegen eine Alternativhypothese $H_A : \theta \in \Theta_A$ testen. Eine Familie von Tests $(T, (K_t)_{t \geq 0})$ heisst \textbf{geordnet} bzgl. $T$ falls $K_t \subset \R$ und $s \leq t \Rightarrow K_t \subset K_s$ gilt. Typische Beispiele dafür sind $K_t = (t, \infty)$ (rechtsseitiger Test), $K_t = (-\infty, -t)$ (linksseitiger Test) und $K_t =  (-\infty, -t) \cup (t, \infty)$ (beidseitiger Test).

\begin{mainbox}{p-Wert}
    Sei $H_0: \theta = \theta_0$ eine einfache Nullhypothese und $(T, K_t)_{t \geq 0}$ eine geordnete Familie von Tests. Der \textbf{$p$-Wert} ist definiert als ZV $G(t)$, wobei:  
    $$G : \R_+ \mapsto [0,1], \quad G(t) = \Pm_{\theta_0}[T \in K_t]$$
\end{mainbox}

Der $p$-Wert hat folgende Eigenschaften:
\begin{subbox}{}
\begin{enumerate}
    \item Sei $T$ stetig und $K_t = (t, \infty)$, so ist der $p$-Wert unter $\Pm_{\theta_0}$ auf $[0,1]$ gleichverteilt.
    \item Für einen $p$-Wert $\gamma$ gilt, dass alle Tests mit Signifikanzniveau $\alpha > \gamma$ die Nullhypothese verwerfen.
\end{enumerate}
\end{subbox}

Wir können zusammenfassend sagen:
\begin{subbox}{}
    $$p\text{-Wert ist klein } \implies H_0 \text{ wird wahrscheinlich verworfen}$$
\end{subbox}

