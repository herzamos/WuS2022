\section{Zufallsvariablen}

Eine \textbf{Zufallsvariable} ist eine messbare Abbildung $X : \Omega \mapsto \R$, wobei $\Omega$ die Ereignismenge eines Wahrscheinlichkeitsraums $(\Omega, \F, \Pm)$ ist.
$$\forall x \in \R. \quad \{\omega \in \Omega \; | \; X(\omega) \leq x\} \in \F.$$
Hierbei schreiben wir oftmals nur $X$.


\subsection{Verteilungsfunktion}

Die \textbf{Verteilungsfunktion} ist die Abbildung $F_X:\R \mapsto [0,1]$ definiert durch:
$$\forall x \in \R, \; F_X(x) = \Pm[X \leq x]$$

\begin{mainbox}{} Aus $a < b$ folgt:
$$\Pm[a < X \leq b] = F_X(b) - F_X(a)$$ \end{mainbox}

\begin{subbox}{Verteilungsfunktion Eingeschaften}
Die Verteilungsfunktion $F$ hat folgende Eigenschaften:
\begin{enumerate}
    \item $F$ ist monoton wachsend
    \item $F$ ist rechtsstetig, d.h. $\lim_{t \to 0} F(x + t) = F(x)$.
    \item $\lim_{x \to - \infty}F_X(x) = 0$ und $\lim_{x \to \infty}F_X(x) = 1$
\end{enumerate}
\end{subbox}

\subsection{Unabhängigkeit von ZV}

\begin{mainbox}{Unabhängigkeit von ZV}
Die ZV $X_1,...,X_n$ sind unabhängig falls:
\begin{align*}
    \forall x_1,...,x_n \in \R. \quad &\Pm[X_1 \leq x_1, ..., X_n \leq x_n] \\ &= \Pm[X_1 \leq x_1] \cdot ... \cdot \Pm[X_n \leq x_n]
\end{align*}
\end{mainbox}

Eine Folge von Zufallsvariablen $X_1, X_2, ...$ ist:
\begin{enumerate}
    \item unabhängig, falls $\forall n. \quad  X_1,...,X_n$ unabhängig sind
    \item unabhängig und identisch verteilt (uiv.), falls sie unabhängig sind und $\forall i,j. \quad F_{X_i} = F_{X_j}$
\end{enumerate}


\subsection{Transformation von ZV}

Sei $\varphi: \R \mapsto \R$ und $X$ ein Zufallsvariable, so ist 
$$\varphi(X) = \varphi \circ X$$ 
auch eine ZV. Seien $X_1,...X_n$ ZV mit $\phi: \R^n \mapsto \R$, so ist 
$$\phi(X_1, ..., X_n) = \phi \circ (X_1,...,X_n)$$
ebenfalls eine ZV.


%------------
% \columnbreak
%------------


\subsection{Konstruktion einer ZV}

Sei eine gültige Verteilungsfunktion $F_X$ gegeben, nun wollen wir eine dazugehörige ZV $X$ konstruieren. Dafür brauchen wir: \medskip

\begin{mainbox}
    {Kolmogorov Theorem}
    $\exists (\Omega, \F, \Pm)$ und $\exists X_1, X_2, ... \text{ ZV in } (\Omega, \F, \Pm)$ sodass $X_1, X_2, ...$ uiv. Bernoullivariablen mit $p = 0.5$ sind.
\end{mainbox}

Sei $X_1, X_2,... \sim \text{Ber}(1/2)$ eine unendliche Folge, dann ist 
$$U = \sum_{n = 1}^\infty 2^{-n}\cdot X_n$$ 
gleichverteilt auf $[0,1]$. \medskip

Aufgrund der Eigenschaften der Verteilungsfunktion $F$, wissen wir dass eine eindeutige Inverse $F^{-1}$ existiert. wir können die generalisierte Inverse definieren als: 
\begin{subbox}{Die generalisierte Inverse}
$\forall \alpha \in [0,1]. \quad F^{-1}(\alpha) = \inf \{x \in \R \; | \; F(x) \geq \alpha\}$
\end{subbox}

Sei nun $F$ eine Verteilungsfunktion und $U$ eine gleichverteilte ZV in $[0,1]$. Dann besitzt $X = F^{-1}(U)$ genau die Verteilungsfunktion $F_X = F$. 